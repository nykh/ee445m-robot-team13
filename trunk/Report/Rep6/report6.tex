\documentclass[a4paper]{article}

\usepackage{fullpage}

\usepackage{graphicx}
\usepackage{caption}
\usepackage{subcaption}

\usepackage{amsmath}

% Code for typesttting C code
\usepackage{listings}
\usepackage{color}

\definecolor{mygreen}{rgb}{0,0.6,0}
\definecolor{mygray}{rgb}{0.5,0.5,0.5}
\definecolor{mymauve}{rgb}{0.58,0,0.82}
\definecolor{mybrown}{rgb}{0.5,0,0}

\lstdefinestyle{MyCStyle}{ %
  language=C,                      % the language of the code
  backgroundcolor=\color{white},   % choose the background color; you must add \usepackage{color} or \usepackage{xcolor}
  basicstyle=\ttfamily    ,        % the size of the fonts that are used for the code
  breakatwhitespace=false,         % sets if automatic breaks should only happen at whitespace
  breaklines=true,                 % sets automatic line breaking
  captionpos=b,                    % sets the caption-position to bottom
  commentstyle=\color{mygreen},    % comment style
  deletekeywords={...},            % if you want to delete keywords from the given language
  escapeinside={\%*}{*)},          % if you want to add LaTeX within your code
  extendedchars=true,              % lets you use non-ASCII characters; for 8-bits encodings only, does not work with UTF-8
  frame=single,                    % adds a frame around the code
  keepspaces=true,                 % keeps spaces in text, useful for keeping indentation of code (possibly needs columns=flexible)
  keywordstyle=\color{blue},       % keyword style
  morecomment=[l][\color{mybrown}]\#,  % compiler directive
  morekeywords={*,...},            % if you want to add more keywords to the set
  numbers=left,                    % where to put the line-numbers; possible values are (none, left, right)
  numbersep=5pt,                   % how far the line-numbers are from the code
  numberstyle=\tiny\color{mygray}, % the style that is used for the line-numbers
  rulecolor=\color{black},         % if not set, the frame-color may be changed on line-breaks within not-black text (e.g. comments (green here))
  showspaces=false,                % show spaces everywhere adding particular underscores; it overrides 'showstringspaces'
  showstringspaces=false,          % underline spaces within strings only
  showtabs=false,                  % show tabs within strings adding particular underscores
  stepnumber=1,                    % the step between two line-numbers. If it's 1, each line will be numbered
  stringstyle=\color{mymauve},     % string literal style
  tabsize=2,                       % sets default tabsize to 2 spaces
  title=\lstname                   % show the filename of files included with \lstinputlisting; also try caption instead of title
}

\newlength{\pic}

\begin{document}

\title{EE445M Lab 6 Report}
\author{\bfseries Yen-Kai Huang, Siavash Zanganeh Kamali, Chen Cui, Miao Qi, Yan Zhang}
\maketitle

\section{Objective} The goal of this lab is to prepare for the final Robot race. In this lab we will interface various
components, including Ping)))) ultrasonic distance sensor, IR distance sensor, and DC motor. A layered communication system
using CAN protocol will transmit the information between two microcontrollers. We will design and implement a software
communication protocol to transmit the sensor data and other things.

In this lab we will also form a team of 4 or 5, which requires us also to apply communication skills to function as a team.

\section{Hardware Design}

\section{Software Design} 

\section{Measurement}

\section{Analysis}

\paragraph{(1) What is one advantage of the Ping))) sensor over the GP2Y0A21YK sensor? \\ }

Ping))) outputs a PWM signal which duty cycles changes linearly with respect to the distance, so it's easier to convert the raw data to distance.

\paragraph{(2) What is one advantage of the \emph{2Y0A21YK} sensor over the Ping))) sensor \\ }

\emph{GP2Y0A21YK} measures the distance by an analog voltage. It is easier to convert voltage to digital data using ADC compared to measure time in Ping.

\paragraph{(3) Describe the noise of the \emph{GP2Y0A21YK} when measured with a spectrum analyzer. \\ }

The noise of the sensor is periodic square waves. An square wave contains all different frequencies, with Maximum amplitude at the frequency of the square wave.
You can see that the spectrum detects many different frequencies.

\paragraph{(4) Why did you choose the digital filters for your sensors? \\ }
What is the time constant for this filter? I.e., if there is a step change in input, how long until your output changes to at least 1/e of the final value?

Since the transition bandwidth of the analog filter is too large for a 2-pole low-pass filter, therefore we use the digital filter to filter out the noise. 

An analog filter with higher poles is more expensive. 

Since we're using a 51 point FIR filter, and $1/e = 0.36$, we need to sample about 17 points to get enough data to represent 0.36 of the input. 

Therefore, $\text{time constant } = \frac{17}{f_c} = 8.5 \, ms$

\paragraph{(5) Present an alternative design for your H-bridge and describe how your H-bridge is better or worse? \\ }

Alternate design :

\paragraph{(6) Give the single-most important factor in determining the maximum bandwidth on this distributed system.
Give the second-most important factor. Justify your answers.  \\ }

Since the system bandwidth was much less than the most possible theoretical bandwidth according to bit rate, the most important factor is the amount of time needed to prepare a package to be sent in the CAN driver, and the amount of time to receive and interpret an incoming package.

The second-most important factor is the bit-rate that determines the amount of time it takes for a single package to be transmitted over the bus.

\section{Post-Mortem Team Evaluation}

\end{document}
